\chapter{Thrust 3: Metric Translation}\label{chap:thrust3}

The standard output of \glspl{nfcs}, in general, consists of material
inventories and flow, each with a composition that may change over time, and
possibly facility deployment histories including time varying capacity
factors.  Most users of these tools, however, are interested in quantities
that are derived from these fundamental data.  In the simplest case, such as
decay heat and radiotoxicity, the metrics of interest may be easily derived
from the material inventories and flows using tabulated data.  In more
complicated cases, however, it may be necessary to perform more complex
post-processing.  Many socioeconomic metrics, including economic analysis,
require this kind of approach.

Thrust 3 was led by Paul Wilson at the University of Wisconsin-Madison, to
provide a mechanism to translate the fundamental data stored by \Cyclus into
metrics of interest to a broad set of users.  This thrust was the most
impacted by the change in scope to include Thrust 0, and only a basic
demonstration of this capability was accomplished.

Since the output data from \Cyclus is even more obscure than the material
inventories and flows common in other tools, the first step in this thrust was
to make that conversion.  With that information availabe, an extensible tool
was developed that allowed users and analysts to introduced new complex
metrics by defining their dependency on simpler metrics.  The Cymetric tool,
described below, supported the automatic resolution of metric dependency all
the way to the fundamental simulation data.  This thrust also informed a
handful of metric translations for inclusion in Cyclist as described in
Chapter \ref{chap:thrust4}.  Finally, some exploratory work for economic
analysis of transitions was completed using the Cymetric tool.

\section{Fundamental Data}

The fundamental output of \Cyclus is optimized to correspond to the discrete
data and discrete facility paradigm: a list of material transactions as a
function of time in which the composition of each material object is stored
separately in order to reuse compositions where appropriate.  While this
format leads to more compact output databases from \Cyclus simulations, it is
not a natural way to analyze fuel cycle performance.

Converting a stream of transactions to inventories and flows is relatively
straightforward.  Conceptually, each transaction that involves a given
facility represents a material addition or subtraction to the inventory at
that facility.  A separate database query is necessary to access to
composition of that transaction, or possibly multiple compositions if multiple
discrete objects were traded.  Combining the quantity and composition of each
object involved in the transaction allows the oveall composition of the
inventory to be updated.

For some facilities, material objects are created and/or destroyed according
to the physics models that they represent.  These actions are recorded in
separate database tables that must also be tallied to get an accurate
accounting of the inventories in any facility at any point in the simulation.

This capability was implemented in a stand-alone tool known as Cyan.  In
addition to deriving the time-dependent inventories in each facility, it
tallied cumulative flows among facilities and can produce graphical
representations of the fuel cycle indicating the flows of material along the
commodity arcs between facilities.

After experience with \Cyclus on a number of problems, it became clear that
the complexity of performing this operation robustly during post-processing
was overly burdensome compared to the cost of extending the fundamental
\Cyclus database to contain this information.  An option was added to the
\Cyclus command-line tool to automatically accummulate the inventories in each
facility as a function of the simulation time.

\section{Cymetric}

In the spirit of the extensible nature of the \Cyclus ecosystem, the
post-processing tools were also designed to be extensible.  This extensibility
is valuable for possible future metrics that are derived from the generic data
stored in a \Cyclus output database, as well as for metrics that might be
based on tables that are associated with a particular archetype.


\subsection{Derived Metric Framework}

\subsection{Sample Derived Metrics}

\section{Economics}
