\section*{Executive Summary}

This project made substantial progress on its original aim for providing a
modern user experience for nuclear fuel cycle analysis while also creating a
robust and functional \acrlong{ngfcs}.

The \Cyclus kernel experienced a dramatic clarification of its interfaces and
data model, becoming a full-fledged agent-based framework, with strong support
for third party developers of novel archetypes.  The most important
contribution of this project to the the development of \Cyclus was the
introduction of tools to facilitate archetype development.  These include
automated code generation of routine archetype components, metadata
annotations to provide reflection and rich description of each data member's
purpose, and mechanisms for input validation and output of complex data.

A comprehensive social science investigation of decision makers' interests in
nuclear fuel cycles, and specifically their interests in \glspl{nfcs} as tools
for understanding nuclear fuel cycle options, was conducted.  This included
document review and analysis, stakeholder interviews, and a survey of decision
makers.  This information was used to study the role of visualization formats
and features in communicating information about nuclear fuel cycles.

A flexible and user-friendly tool was developed for building \Cyclus analysis
models, featuring a drag-and-drop interface and automatic input form
generation for novel archetypes.  Cycic allows users to design fuel cycles
from arbitrary collections of facilities for the first time, with mechanisms
that contribute to consistency within that fuel cycle.  Interacting with some
of the metadata capabilities introduced in the above-mentioned tools to
support archetype development, Cycic also automates the generation of user
input forms for novel archetypes with little to no special knowledge required
by the archetype developers.

Translation of the fundamental metrics of \Cyclus into more interesting
quantities is accomplished in the Cymetric python package.  This package is
specifically designed to support the introduction of new metrics by building
upon existing metrics.  This concept allows for multiple dependencies and
encourages building complex metrics out of incremental transformations to
those prior metrics.  New archetype developers can contribute their own
archetype-specific metric using the same capability.  A simple demonstration
of this capability focused on generating time-dependent cash flows for reactor
deployment that could then be analyzed in different ways.




