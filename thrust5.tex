\chapter{Thrust 5: Efficient Design of a Client-Server Model}\label{chap:thrust5}

One goal for a \gls{ngfcs} is deployment on a variety of computing platforms,
including those with greatly reduced resources (mobile/tablet).  This suggets
a client-server approach in which the simulation itself occurs on a remote
execution server and only the visualization occurs on a local device.  To
ensure that the user experience meets expectations on the local device, the
software must be designed to accommodate, as much as possible, the limitations
of the network connection.  Another advantage of a client-server strategy is
the ability to scale up the computing resource on the server side, as
necessary for larger and more complex problems.

Thrust 5 was led by Robert Hiromoto at the University of Idaho, to understand
the network requirements and limitations of deploying a client-server
computational model.  Like some of the other thrusts, this task was
undermined, in part, by the lack of a fully functioning \gls{ngfcs}.  Since
this thrust also depended on a demonstration of the visualization software
that would run locally, that was itself delayed by the lack of a \gls{ngfcs}
at the projects start, it was largely reformulated.

The outcome of Thrust 5 was a lightweight server that would listen for \Cyclus
tasks, perform the simulation, and offer the output database for further
analysis.  This capability was deployed in two related capacities: as a remote
execution server for \Cyclus users and as a compute farm for large scale
analysis.  This became a key component to providing platform-independent
access to \Cyclus when coupled with the Cyclist tool for model building and
data visualization.

Cloudlus\citeprod{carlsen_cloudlus_2014} is a lightweight wrapper, written in
the Go programming language, that listens on an internet port for tasks,
performs those tasks by launching a \Cyclus simulation, and then serves the
output database on the same internet port.  Cloudlus has been deployed
continuously since 2014 as the mechanism for users to submit \Cyclus jobs
through the website, www.fuelcycle.org.  The same Cloudlus service is
available to users of Cyclist as a remote execution service as described
briefly in Chapters \ref{chap:thrust2} and \ref{chap:thrust4}.

Cloudlus also supports a mode for large scale deployment and analysis.  In
this mode, many Cloudlus workers are deployed in cloud computing resources and
register with a single master process that then issues requests to those
workers. The master can be designed/configured to generate that work in many
different ways. Cloudlus has been deployed on commercial cloud resources and
UW-Madison's \gls{HTC} computing system to support large scale
optimization\citeref{chtc}.  In this case, the master implemented an
optimization algorithm that relied on 1000's of independent \Cyclus
simulations, with that master automatically determining the input file for
each simulation based on the results of previous simulations.

This infrastructure has been sufficient, to date, but may not scale well as
problems become more complex due to the size of the output database.  Storage
and network transfer of those databases may become a challenge as more complex
problems result in databases of many gigabytes.  It will continue to be
sufficient for beginner use through the website or the Cyclist interface, but
as more robust and scalable replacements for Cloudlus emerge, better remote
execution paradigms are also expected to merge.
