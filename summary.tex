\chapter{Summary}

This project made substantial progress on its original aim for providing a
modern user experience for nuclear fuel cycle analysis while also creating a
robust and functional \acrfull{ngfcs}.

Early changes in project scope required more investment in the generation of a
\gls{ngfcs} that originally planned, delaying progress in some of the thrust
areas and ultimately preventing the project from being as tightly integrated
as intended.  One of the most important aims that was not accomplished was a
demonstration of substantially different user experiences for different
categories of stakeholders.  Most thrusts areas scaled back the number of
stakeholder groups to a single group, most often a sophisticated nuclear
engineering stakeholder group.  While Thrust 1 did provide insight into the
interests of a different stakeholder group, namely national-scale decision
makers, there was no opportunity to demonstrate a user experience that suited
this group.

The success of the \Cyclus kernel did catalyze a growing community of
contributors to an active ecosystem.  A number of institutions have since been
funded to contribute an array of alternative archetypes for facilities
including enrichment facilities, fuel fabrication facilities and reactors.
Additional investments were made in optimization efforts related to \Cyclus
including the detailed development of the \acrfull{DRE} and an external
optimization wrapper that was able to identify an optimal transition strategy.

Future efforts should focus on further developing post-processing capability
across a variety of different stakeholder groups.  While there is constant
demand for more visualization capabilities for advanced users, there continues
to be value in the original premise of providing different visualizations to
different audiences, based on the same simulation kernel.

In addition, it will be valuable to see \Cyclus applied to real fuel cycle
analysis.  This kind of application is critical to all aspects of continued
progress in the \Cyclus ecosystem.  Most importantly, it will uncover and help
prioritize important needs in the user experience.  It will undoubtedly also
lead to the development of new archetypes that are customized to suit the
particular fuel cycle problem being solved.  Through these developments, the
community will identify common improvements, needs and directions for the
ecosystem as a whole.
