\chapter{Introduction}

Interest in \glspl{nfcs} by the \gls{USDOE} Office of Nuclear Energy intensified
in the mid-2000's as a variety of policy issues converged to prompt a renewed
interest in advanced nuclear fuel cycles.  Whereas closed fuel cycles have
always been imagined as a way to extend uranium resources, their ability to
reduce the burden of radioactive waste management was a new motivation, as was
the possibility of reducing global proliferation risk by implementing
international nuclear fuel cycle services.  With a richer set of possible
benefits, new tools were necessary to provide insight into the degree to which
different technology options would have an impact on these and other
socio-political metrics.

The fundamental purpose of a \gls{nfcs} is to track the mass flows and
isotopic composition of material as it moves among facilities in a complete
nuclear fuel cycle.  At a minimum, the set of facilities should include
nuclear reactors for the consumption of uranium and the production of
transuranics and fission products, but most tools include front-end
facilities including mines, enrichment facilities and fuel fabrication
facilities, as well as back-end facilities including storage and separations.
As they become more sophisticated, a \gls{nfcs} will also track information
about facility deployment, utilization and decommissioning, in order to
support economic assessments that are dominated by the capital costs and
capacity factors of those facilities.

A variety of such tools emerged, often developed to answer specific
programmatic questions.  Most notable among these tools were
Dymond\citeref{yacout_modeling_2005}, DANESS\citeref{van_den_durpel_daness_2009} and VISION\citeref{jacobson_vision_2009} all
initially developed at national laboratories and sharing a common heritage.
Although they continue to provide value to the nuclear fuel cycle analysis
community, a number of limitations were identified in these tools, leading to
a desire for a \gls{ngfcs}.

One limitation of the previous tools was their reliance on a systems dynamics
framework.  Such frameworks were initially considered attractive because of
the graphical development environments of stocks and flows that mapped well
onto the material flow modeling at the core of a \gls{nfcs}.  However, the
stock and flow paradigm requires all models to be expressed in the context of
a first-order ODE solved with an Eulerian approach.  Many models are not
well-suited to this approach and require the implementation of alternative
methods as extensions to the systems dynamics framework using a more
traditional programming interface, thus diminishing the benefit of the graphical
development environment.  Systems dynamics platforms also suffered from
limitations on the size of problem they could accommodate, particularly early
in the development of these tools.  As it became necessary or desirable to
expand the number of isotopes tracked in material flows and incorporate large
matrices to represent the long term response of those material flows, limits
in the total data footprint were reached.  \glspl{nfcs} were often the largest
models to ever be implemented in these platforms and close collaboration with
their developers led to improvements that relaxed these limits.  The systems
dynamics platforms were not well-suited to collaborative development because
contributions from individual developers could only be merged manually.
Finally, these platforms required the purchase of commercial licenses for each
user at each site, both costly and an obstacle to deployment at large scales
for parametric studies, sensitivity analysis and/or optimization.

Another limitation arose from the problem-focused development path for these
tools.  Because it was necessary to prioritize particular candidate fuel
cycles, they each developed depth in their modeling fidelity to support those
specific fuel cycles and without sufficient modularity to quickly incorporate
novel technologies.  As policy-driven interest in specific fuel cycles waned,
demand grew for these tools to analyze the fuel cycle impact of a variety of
novel nuclear energy technologies.  Incorporating these technologies in the
existing tools required a steep learning curve to first learn the particular
systems dynamics platform and then understand the modeling approaches used to
incorporate the complexity of real fuel cycles.  This was often beyond the
scope of many small technology development efforts, especially when crude
estimates of the steady-state fuel cycle impacts were possible with simple
spreadsheet models.  At the same, time each of the tools had implemented
slightly different assumptions and approximations rendering comparison a
challenge and ultimately undermining confidence in their ability to produce
reliable results.

Finally, the user experience for these tools was never a focus during their
development, despite the recognized value in enabling their use by a broader
set of audiences to improve the communication of their results.  The graphical
nature of the systems dynamics platforms was considered to be sufficient until
the implemented models became increasingly complex.  Fuel cycle scenarios were
defined by data in ever-growing spreadsheets and results stored in even bigger
spreadsheets.  Visualization of both input and output data was generally
limited to the spreadsheet software's capabilities.

Therefore, the Fuel Cycle Research \& Development program chose to support the
development of a \gls{ngfcs} platform with sufficient modularity to
accommodate a wide variety of audiences, use cases and developer needs. The
requirements for this framework included the modeling of discrete transactions
of materials among a set of discrete facilities operating in distinct
geographic regions. All transactions would arise as the result of resolving a
set of offers and requests in a market whose transaction costs may be
determined by a wide array of economic, political and technical
parameters. Most importantly, a modular “plug-in” architecture would allow
user-developers to introduce different models with alternative market and
facility behaviors, thereby isolating variations in modeling to single
changes. The fundamental data generated by such a framework includes a list of
material transactions and the facilities involved in such transactions, a
history of the isotopic composition of each material object, and an operating
history for each facility. As additional plug-in modules are developed,
additional data will be generated related to the facility histories, material
histories, or market transactions.

The NGFCS was also expected to be a useful evaluation tool for a variety of
audiences.  Non-technical audiences would interact with the NGFCS in a way
that allowed them to express critical high-level decisions and understand the
outcomes of those decisions in a set of key metrics. At the same time, expert
audiences might be interested in a quantitative technology assessment to help
motivate design improvements. Developers of the NGFCS would need to visualize
their results to ensure consistency and correctness. Given the wide array of
possible metrics and summaries, each type of user might want to gradually
explore the results with increasing depth and sophistication. At the same
time, it was important that the NGFCS allow experts to introduce new modules
to capture specific physical models or market behavior, but rely on a common
infrastructure that facilitates direct comparisons of results.

\section{Research Plan Overview}

This project was therefore designed to explore strategies and develop tools to
enhance the user experience for the \gls{ngfcs} for a range of target
audiences.  The research plan identified five thrusts:
\begin{enumerate}
  \item Stakeholder, parameter and metric identification
  \item User interface and model generation
  \item Metric translation
  \item Visualization environment
  \item Efficient design of a client-server model
\end{enumerate}

The work scope definition for this project referred to a \gls{ngfcs} platform
that would be developed by the \gls{USDOE} and requested proposals for
enhancements to that platform, explicitly prohibiting proposals that aimed to
develop the platform itself.  However, at the time that this project was
awarded, no such platform existed and there were no plans to create one.
Given ongoing efforts to design and implement such a platform at the lead
institution, a renegotiated scope and research plan was developed, inserting
an additional Thrust 0 that would dedicate resources to the completion of the
\Cyclus framework as the \gls{ngfcs}.  The addition of this thrust required
some scaling back of the scope of each of the other thrusts with the ultimate
research plan outlined below.

Thrust 0 was lead by nuclear engineering software developers to finalize and
implement the design of \Cyclus.  While most capabilities had been
demonstrated in some fashion, this project allowed for robustness and
stability improvements that enable a broader adoption and simplifying the
process of extending the platform with new models.

Thrust 1 was led by experts in communication science, and specifically in the
science of communicating about controversial science.  This critical project
element was designed to enhance the relevance of the \gls{ngfcs} to a wider
audience than the traditional nuclear engineering audience of previous tools
by identifying the metrics that address the questions and concerns of those
audiences rather than focusing on the metrics that the nuclear engineering
community values.  This thrust incorporated a thorough review of prior
testimony and historical documents to identify stakeholders and the issues
that they raised, interviews with a variety of individuals representing those
stakeholder groups to further identify the important issues, a survey of a
larger community of stakeholder to develop a quantitative understanding of how
those issues were related to each other, and an experiment to understand how
different forms of visualization would communicate the results of a fuel cycle
simulation.  A detailed account of the work performed under this thrust is
found in Chapter \ref{chap:thrust1}.

The primary purpose of Thrust 2 was to develop a convenient graphical
interface by which users could define the fuel cycle scenarios of interest to
them.  Inspired by the drag-and-drop promise of the systems dynamics paradigm,
this interface would allow users to build high-level graphical representation
of a nuclear fuel cycle by placing and connecting individual facilities.
Particular consideration would be given to the rules for connecting facilities
to reduce the depth of nuclear engineering domain knowledge expected of users
while still preserving a physically valid flow path for material.  The
interface would expose varying level of detail, with many default values
hidden from users seeking a simpler experience, but available to those users
looking to control all parameters.  Each of these parameters would be
self-documented by the interface allowing users to quickly learn what the
valid purpose and range of each parameter is. Chapter \ref{chap:thrust2}
outlines the development at testing of such an interface.

Thrust 3 focused on translating the fundamental data recorded by the
\gls{ngfcs} into metrics identified by Thrust 1 as interesting to
stakeholders.  Since the \gls{ngfcs} would operate at the fidelity of discrete
material objects, perhaps individual fuel assemblies, being transacted by
discrete facilities, a minimum amount of data would be recorded directly in
its fundamental output.  This data might include, for example, the masses and
compositions of individual material objects being transacted.  Although some
applications may be interested in such high-fidelity data, many applications
would be integrated in integrated quantities such as total flow rates and or
inventories in the fuel cycle.  Moreover, there may be interest in quantities
derived from this aggregate data such as decay heats or radiotoxicity.
Finally, given the role of modularity and extensibility in the \gls{ngfcs}
itself, it is important for this metric translation capability to offer
similar extensibility.  Chapter \ref{chap:thrust3} describes some sample
metrics and the conceptual framework embodied in an extensible metric
translation capability.

The ability for user to explore their data in an organic and discovery-driven
manner was the goal of Thrust 4.  Beyond simply providing simple plots to
represent the output metrics (identified in Thrust 1 and calculated in Thrust
3), the user may want to aggregate some results over select dimensions and
filter those results across other dimensions.  When those features are shared
over a set of related visualizations, they form a workspace in which all the
results can be manipulated as a set.  Chapter \ref{chap:thrust4} describes the
implementation of these concepts as well as a custom visualization for the
flow of material among different fuel cycle components.

Although it was not clear how much computational power would be necessary to
run a single case with the \gls{ngfcs}, it was likely that certain use cases
would require remote operation in order to provide sufficient computing
capability, particularly if thin clients such as tablets are imagined as
ultimate user interface devices.  Thrust 5 considered the factors that would
facilitate a client-server mode of operation in which the simulation itself
was operating remotely and only the visualization tools were local to the PC,
tablet, or smartphone.  This resulted in the development of a cloud-like
simulation capability that can support remote execution of single simulations
submitted via a website or with the graphical interface developed in Thrust
2, as well as large scale analysis by deployment as a compute farm.


